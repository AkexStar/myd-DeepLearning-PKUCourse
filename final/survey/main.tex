\documentclass[conference,compsoc]{IEEEtran}

% IEEEtran journal, technote, conference, peerreview, peerreviewca
% compsoc, transmag

% some very useful LaTeX packages include:

\usepackage{ctex}
\usepackage{graphicx}
\usepackage{animate}
\usepackage{subfigure}
\usepackage{url}
\usepackage{amssymb}
\usepackage{amsmath}
\usepackage{cite}
\usepackage{multirow}
\usepackage{listings}
\usepackage{booktabs}
\usepackage[colorlinks,linkcolor=blue]{hyperref}
\usepackage{listings}
\usepackage{xcolor}
\usepackage{float}
\usepackage{hyperref}
\usepackage{caption}
\usepackage{listings}
\usepackage{subfigure}
\usepackage{stfloats}

% Define code style
\lstset{
    columns=fixed,
    frame=none,
    keywordstyle=\color[RGB]{40,40,255},
    numberstyle=\footnotesize\color{darkgray},
    commentstyle=\it\color[RGB]{0,96,96},
    stringstyle=\rmfamily\slshape\color[RGB]{128,0,0},
    showstringspaces=false,
    tabsize=4
}

% Replace with terms in Chinese
\renewcommand{\abstractname}{摘要}
\renewcommand{\IEEEkeywordsname}{关键词}
\renewcommand{\figurename}{图}
\renewcommand{\tablename}{表}
% Change the style of abstract and keywords
\renewcommand{\abstract}{{\centering\section*{\abstractname}}}
\renewcommand{\IEEEkeywords}{\vspace{2ex} \hspace{-2em}\textbf{\IEEEkeywordsname}: \vspace{2ex}}

% correct bad hyphenation here
\hyphenation{op-tical net-works semi-conduc-tor}

% Your document starts here!
\begin{document}

% Define document title and author
\title{\LARGE{Gaussian Process}}
% \author{李锦韬 2201213292 lijintao@stu.pku.edu.cn}
\author{\IEEEauthorblockN{
李锦韬
}
\IEEEauthorblockA{\upshape 学号:2201213292}
Email: lijintao@stu.pku.edu.cn}
\maketitle

% Write abstract here
\begin{abstract}
    过程回归是基于贝叶斯理论和统计学习埋论发展起来的一种全新机器学习方法‚适于处理高维数、小 样本和非线性等复杂回归问题在阐述该方法原理的基础‚分析了其存在的计算量大、噪声必须服从高斯分布等 问题‚给出了改进方法与神经网络和支持向量机相比‚该方法具有容易实现、超参数自适应获取以及输出具有概率 意义等优点‚方便与预测控制、自适应控制、贝叶斯滤波等相结合最后总结了其应用情况并展望了未来发展方向
\end{abstract}
% Write keywords here
\begin{IEEEkeywords}
    对抗攻击,黑盒攻击,白盒攻击
\end{IEEEkeywords}


% Main Part
\section{\textbf{问题背景}}
神经网络和支持向量机相比‚方法具有容 易实现、灵活的非参数推断、超参数自适应获取等优 点‚是一个具有概率意义的核学习机‚可对预测输出 做出概率解释‚在实际应用中已取得了许多令人满意 的成果但是‚目前方法还不够完善‚仍在不断地 发展‚主要有以下几个发展趋势‚
\subsection{问题与展望}
寻求效率更高的协方差求逆计算方法或训练集选择 方法仍是不变的研究内容一方面‚可以结合计算机 软硬件及并行计算技术‚提高计算效率另一方面‚自 动处理数据并寻找“信息数据”以压缩数据集来降低 计算量是另一发展趋势此外‚基于模型的递归 辨识或在线学习方法的高效实现方法仍面临着一些 挑战 对于控制系统而言‚抗扰性能至关重要‚但 是目前大部分基于模型的控制方法更多地仅关 注设定点的跟踪性能‚缺少关于抗干扰的性能分析和 设计·

\begin{table}[htbp]
  \caption{表範例}
  \begin{center}
  \begin{tabular}{ccccc}
  \hline
  \textbf{Mar 2016} & \textbf{Mar 2015} & \textbf{Change} & \textbf{Language} & \textbf{Ratings} \\
  \hline
  1 & 2 & $\Uparrow$   & Java   & 20.53\% \\
  \hline
  2 & 1 & $\Uparrow$   & C      & 14.60\% \\
  \hline
  3 & 4 & $\Downarrow$ & C++    & 6.72\%  \\
  \hline
  4 & 5 & $\Uparrow$   & C\#    & 4.27\%  \\
  \hline
  5 & 8 & $\Uparrow$   & Python & 4.26\%  \\
  \hline
  6 & 6 &              & PHP    & 2.77\%  \\
  \hline
  \end{tabular}
  \label{tab1}
  \end{center}
  \end{table}

另外‚基于模型的鲁棒控制设计也将是今 后研究的趋势之一 利用方法辨识动态系统的状态方程和观 测方程‚有效解决了滤波过程中由于模型不准确或统 计特性未知而导致滤波结果发散的问题‚优势明显‚ 
\medskip

可以与更多滤波方法如容积卡尔曼滤波等相结合‚ 并应用于实际工程中 随着贝叶斯理论和统十十学习理论的进一步深入 发展以及计算技术的飞速进步‚日趋成熟完善和不断 实用化的方法将不断拓宽其应用领域‚如生物 系统等不确定系统模型辨识等而新应用、新要求也 将促使方法不断取得新的进展

\section{fd}
一个模型是最好的‚或者一个模型比另一个模型更 好一种方法是寻找一组能使某一损失函数二最 小化的参数通常采用的损失函数为二次损失函数‚ 

\begin{equation}
  UFC = \sum_{i=1}^{5} \sum_{j=1}^{3} N_{ij} W_{ij} \label{eq1}
  \end{equation}

典型的例子有最小乘多项式回归、最小乘神 经网络等这种方法存在明显的缺陷仅致力于在训 练集上降低模型误差若为了降低模型误差而一味增 加模型复杂度‚则易导致过拟合‚尽管在训练集上回 归精度较高‚但其泛化能力或预测性能不佳为了避 免过拟合‚可以使用一个相对简单的模型‚它忽略了 复杂特征和噪声‚相对比较平滑但是模型过几简单 也会造成预测性能差另一种方法是极大似然法‚它 不需要损失函数首先由假定的噪声分布得到训练集 的联合概率密度即似然函数‚再通过寻找使似然函 数最大化的参数叨来获得回归模型如果噪声分布满 足高斯分布‚则通过比较似然函数和一几次损失函数不 难发现‚该似然函数的负对数与一次损失函数成一定 比例关系‚因表明了这两种方法在木质上是一样的 为了避免过拟合‚可以采用第类方法‚即贝叶 斯回归该方法定义了一个函数分布‚赋子每一种可 能的函数一个先验概率‚可能性越大的函数‚其先验 概率越大但是可能的函数往往为一个不可数集‚即 有无限个可能的函数随之引入一个新的问题如何 在有限的时间内对这些无限的函数进行选择一种有 效的解决方法即是高斯过程回归 是近年发展起来的一种机器学习回归方法‚ 它有着严格的统计学习理沦基础‚对处理高维数、小 样木、非线性等复杂的问题具有很好的适应性‚且泛 化能力强与神经网络、支持向量机相比‚具有 容易实现、超参数自适应获取、非参数推断灵活以及 输出具有概率意义等优点‚在国外发展很快‚并取得 了许多研究成果‚现已成为国际机器学习领域的研究 热点’一头近几年也逐步得到国内学者的重视‚在许多 领域得到了成功应用阵木文将首先阐述的基 本原理‚对存在的主要问题进行探讨‚总结了相 应的改进方法最后对的应用进行了总结并指 出其未来发展趋势 高斯过程回归原理 预测 从函数空间角度出发‚定义一个高斯过程 来描述函数分布‚直接在函数空间进行贝叶斯推 理‘【‚是任意有限个随机变量均具有联合高斯分 布的集合‚其性质完全由均值函数和协方差函数确定‚

% use section* for acknowledgment
\ifCLASSOPTIONcompsoc
  % The Computer Society usually uses the plural form
  \section*{致谢}
\else
  % regular IEEE prefers the singular form
  \section*{致谢}
\fi
The authors would like to thank...

\begin{thebibliography}{1}

    \bibitem{IEEEhowto:kopka}
    H.~Kopka and P.~W. Daly, \emph{A Guide to \LaTeX}, 3rd~ed.\hskip 1em plus
      0.5em minus 0.4em\relax Harlow, England: Addison-Wesley, 1999.
    
    \end{thebibliography}

\end{document}